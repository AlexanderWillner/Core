% bugfixes ---------------------------------------------------------------------
%\let\spvec\vec                   % llncs with amsmath bugfix
%\let\vec\accentvec               % llncs with amsmath bugfix
%\newcommand{\subparagraph}{}     % IEEETrans with titlesec bugfix

%\makeatletter                    % for old listings package
%\@ifundefined{float@listhead}{}{%
%    \renewcommand*{\lstlistoflistings}{%
%        \begingroup
%            \if@twocolumn
%                \@restonecoltrue\onecolumn
%            \else
%                \@restonecolfalse
%            \fi
%            \float@listhead{\lstlistlistingname}%
%            \setlength{\parskip}{\z@}%
%            \setlength{\parindent}{\z@}%
%            \setlength{\parfillskip}{\z@ \@plus 1fil}%
%            \@starttoc{lol}%
%            \if@restonecol\twocolumn\fi
%        \endgroup
%    }%
%}
%\makeatother
% ------------------------------------------------------------------------------

% macro: ifpackageloaded -------------------------------------------------------
\makeatletter
\providecommand{\IfPackageLoaded}[2]{\@ifpackageloaded{#1}{#2}{}}
\providecommand{\IfPackageNotLoaded}[2]{\@ifpackageloaded{#1}{}{#2}}
\providecommand{\IfElsePackageLoaded}[3]{\@ifpackageloaded{#1}{#2}{#3}}
\makeatother
% ------------------------------------------------------------------------------

% File encoding: Normal UTF8
\IfPackageNotLoaded{CJKutf8}{\usepackage[utf8]{inputenc}}

\usepackage{cmap}                     % Support copy and search for pdflatex
\input{glyphtounicode}                % Support copy and search for pdflatex
\pdfgentounicode=1                    % Support copy and search for pdflatex
\usepackage{amssymb}                  % Springer Verlag (pdflatex)
\usepackage[hyphens]{url}             % Create URLs in the document
\usepackage[
spacing=true,                         % Fonts: spacing?
kerning=true,
tracking=true,                        % Fonts: hyphenatable letterspacing
expansion=true,                       % Fonts: better grey value
protrusion=true]{microtype}           % Fonts: margin kerning
\microtypecontext{spacing=nonfrench}
\usepackage[T1]{fontenc}              % Font encoding: T1
\usepackage{textcomp}                 % For the Euro sign
\usepackage{booktabs}                 % Nice tables
\usepackage{listings}                 % Nice listings
\usepackage{listingsutf8}             % Listings with UTF-8
\lstset{columns=fullflexible,         % Listings in multicolumn mode
        breaklines=true,              % Listings with long lines
        xleftmargin=3em,              % Correct indentation
        basicstyle=\scriptsize,       % Small text
        numbers=left,                 % Show numbers 
        stepnumber=1,                 % For each line
        inputencoding=utf8            % Enable UTF8
        }
\usepackage{algorithmic}              % Nice pseudo code
\usepackage{xcolor}                   % To use and define colors
\definecolor{LinkColor}{rgb}          % Link color
{0.31,0.46,.64}                       %
\definecolor{MarginColor}{rgb}        % Margin color
{0.31,0.46,.64}                       %
\usepackage{marginnote}               % Non floating margin notes
\usepackage{subcaption}               % Subfigures
\usepackage{wrapfig}                  % Use floating images
\usepackage{stfloats}                 % Makes LaTeX honour ‘[b]’ placement
\usepackage{amsmath}                  % For eqref
\usepackage{paralist}                 % For inline lists
\usepackage{fixltx2e}                 % Provides fixes for LaTeX
\usepackage{fix-cm}                   % Provides fixes for LaTeX
\usepackage{rotating}                 % e.g. for the special side margin notes
\usepackage{lipsum}                   % To add some dummy text
\usepackage{xspace}                   % Fixing some spacing issues
\usepackage{ellipsis}                 % Puts space around ellipses
%\usepackage{ragged2e}                 % New commands/env. for setting ragged text
%\usepackage{lmodern}                 % Nicer fonts (for all)
\usepackage{ltxtable}                 % Long complex tables
\usepackage{ltablex}                  % Long complex tables
\usepackage[switch*,modulo]{lineno}   % Show line numbers
\usepackage{blindtext}                % insert blind texts
%\usepackage{cleveref}                 % To ref footnotes twice (use after hyperref)
%\crefformat{footnote}{#2\footnotemark[#1]#3}

\newcommand{\etc}{etc.\@\xspace}      % Fixing some spacing issues
% To be used in the project specific config file:
%\usepackage[bf]{caption}[2008/08/24]  % Bold caption (better contrast)
%\usepackage{acronym}                  % For acronyms
%\usepackage[pdftex]{graphicx}         % Include images for pdflatex

\hyphenation{op-tical net-works semi-conduc-tor con-cept}

\renewcommand{\figurename}{Fig.}
\renewcommand{\tablename}{Tab.}
\newcommand{\sectionname}{Sec.}
\newcommand{\equationname}{Eq.}
\renewcommand{\lstlistlistingname}{List of Listings}

\makeindex

\setcounter{tocdepth}{3}
\newcommand{\mykeywords}[1]{\par\addvspace\baselineskip
\keywordname\enspace\ignorespaces#1}
\makeatletter
\providecommand*{\toclevel@title}{99}
\providecommand*{\toclevel@author}{99}

% type setting ----------------------------------------------------------------
% no "Schusterjungen"
    \clubpenalty = 10000
% no "Hurenkinder"
    \widowpenalty = 10000 
    \displaywidowpenalty = 10000
% type setting ----------------------------------------------------------------

% annotations --------------------------------------------------------------
\newcommand{\annot}[2][]{%
    \pdfannot width \linewidth height 2\baselineskip depth 0pt{%
        /Subtype/Text%
        /Open false
        /Name /Comment%
        /CA .4%
        /C [.3 .6 .9]%
        /T (\pdfescapestring{#1})%            /Contents(\pdfescapestring{\detokenize{#2}})%
    }
}

\newcommand\aclc[1]{%
{%
\renewcommand{\url}[1]{}%
\renewcommand{\footnote}[1]{}%
\renewcommand{\cite}[1]{}%
\let\protect\relax%
\acl*{#1}%
}%
}
\newcommand\acfc[1]{%
{%
\renewcommand{\url}[1]{}%
\renewcommand{\footnote}[1]{}%
\renewcommand{\cite}[1]{}%
\let\protect\relax%
\acf*{#1}%
}%
}
\newcommand\todoremark[1]{%
	\color{blue}%
	((remark: #1))
	\color{black}%
}
\newcommand\todo[1]{%
	\color{red}%
	((todo: #1))
	\color{black}%
}
\newcommand\todotext[1]{%
    \todo{#1}%
	\color[gray]{0.8}%
	\blindtext[1]
	\color{black}%
}
\newcommand\todosmall[1]{%
    \todo{#1}%
	\color[gray]{0.8}%
	Lorem ipsum dolor sit amet, consectetuer adipiscing elit. Etiam lobortis facilisis sem. Nullam nec mi et neque pharetra sollicitudin. Praesent imperdiet mi nec ante.
	\color{black}%
}
\newcommand\todoshort[1]{\todosmall{#1}}
\newcommand\todomid[1]{%
    \todo{#1}%
    \color[gray]{0.8}%
    Lorem ipsum dolor sit amet, consectetur adipisicing elit, sed do eiusmod tempor incididunt ut labore et dolore magna aliqua. Ut enim ad minim veniam, quis nostrud exercitation ullamco laboris nisi ut aliquip ex ea commodo consequat. Duis aute irure dolor in reprehenderit in voluptate velit esse cillum dolore eu fugiat nulla pariatur.
    \color{black}%
}
%\setlength{\marginparwidth}{0.7in}
\newcommand\sidenote[1]{\mbox{}%
  \marginnote{%
    \scriptsize%
    %\raggedright%
    \hspace{0pt}%
    \color{MarginColor}%
    %\begin{turn}{270}%
      \emph{#1}%
    %\end{turn}%
  }%
}%

%for two-column layout
\makeatletter
\g@addto@macro\@mn@margintest{%
  \if@twocolumn
    \ifx\@mn@currxpos\relax% don't know which margin use normal one
      \normalmarginpar
    \else\ifx\@mn@currxpos\@empty% don't know which margin use normal one
        \normalmarginpar
      \else
        \if@tempswa% use \oddsidemargin for tests
          \ifdim\@mn@currxpos >\dimexpr \oddsidemargin+1in+\columnwidth\relax
            \normalmarginpar% right column --> right margin
          \else
            \reversemarginpar% left column --> left margin
          \fi
        \else% use \evensidemargin for tests
          \ifdim\@mn@currxpos >\dimexpr \evensidemargin+1in+\columnwidth\relax
            \reversemarginpar% right column --> right margin
          \else
            \normalmarginpar% left column --> left margin
          \fi
        \fi
      \fi
    \fi
  \fi
}
\makeatother
% annotations --------------------------------------------------------------

% more listings ------------------------------------------------------------
\definecolor{grey}{rgb}{0.5,0.5,0.5}
\lstdefinelanguage{sparql}{
morecomment=[l][\color{black}]{\#},
morestring=[b][\color{black}]\",
morekeywords={SELECT,CONSTRUCT,DESCRIBE,ASK,WHERE,FROM,NAMED,PREFIX,BASE,OPTIONAL,FILTER,GRAPH,LIMIT,OFFSET,SERVICE,UNION,EXISTS,NOT,BINDINGS,MINUS,a},
sensitive=true
}
\lstdefinelanguage{ttl}{
sensitive=true,
morecomment=[l][\color{grey}]{@},
morecomment=[l][\color{black}]{\#},
morestring=[b][\color{black}]\",
}
\definecolor{groovyblue}{HTML}{0000A0}
\definecolor{groovygreen}{HTML}{008000}
\definecolor{darkgray}{rgb}{.4,.4,.4}
 
\lstdefinelanguage{Groovy}[]{Java}{
  keywordstyle=\color{groovyblue}\bfseries,
  stringstyle=\color{groovygreen}\ttfamily,
  keywords=[3]{each, findAll, groupBy, collect, inject, eachWithIndex},
  morekeywords={def, as, in, use},
  moredelim=[is][\textcolor{darkgray}]{\%\%}{\%\%},
  moredelim=[il][\textcolor{darkgray}]{§§}
}
% more listings ------------------------------------------------------------

% own requirement environment --------------------------------------------------
\makeatletter
\newcommand\requirement[3]{%
  \protected@write\@auxout{}{\string\CreateCounter{#1}}%
  \@ifundefined{c@my#1}
    {#1??: #3}
    {\refstepcounter{my#1}\label{#2}#1\arabic{my#1}: #3}%
}
\newcommand{\CreateCounter}[1]{%
  \@ifundefined{c@my#1}
    {\newcounter{my#1}\global\@namedef{themy#1}{#1\arabic{my#1}}}
    {}%
}
\AtEndDocument{\let\CreateCounter\@gobble}
\makeatother
% ------------------------------------------------------------------------------
